
\thispagestyle{empty}

\begin{center}
    \small
    \sc Instituto Federal de Educação, Ciência e Tecnologia de Minas Gerais \\ 
    \sc Campus Formiga \\
    \sc Mestrado Profissional em Administração
    
   
    \vspace{3cm}
    \large {Rodrigo Emiliano dos Santos}
    
    \vspace{3cm}
     \large \textbf{Value Investing no Mercado Brasileiro} \\
   Aplicação da Fórmula de Valor Intrínseco de Benjamin Graham na Seleção e Gestão de Carteiras no Mercado Brasileiro
    
    \vspace{2cm}
    \begin{flushright}
    \begin{minipage}{0.6\textwidth}
    \small
    Projeto apresentado ao Programa de Pós-
    Graduação em Administração do Instituto
    Federal de Educação, Ciência e
    Tecnologia de Minas Gerais - \textit{Campus} Formiga 
    (IFMG - \textit{Campus} Formiga), como requisito para 
    o Exame de Qualificação do Mestrado Profissional em 
    Administração.
    \end{minipage}
    \end{flushright}
    
    \vspace{0.5cm}
    \begin{flushright}
    \small
    Orientador: Prof. Dra Maisa Kely de Melo.\\
    Coorientador: Prof. Dr. Washighton Santos Silva.
    
    \vspace{0.5cm}
    Linha de Pesquisa: Finanças Corporativas e Investimentos.
    \end{flushright}
    
    \vfill
    Formiga, Minas Gerais \\
    2025
\end{center}


\newpage

\thispagestyle{empty}

\newenvironment{meuresumo}{
  \clearpage
  \small
  \vspace{-1cm}
  \begin{center}
    \bfseries RESUMO
    \vspace{0.5em}
  \end{center}
  \begin{quote}
}{
  \end{quote}
  \vspace{-1.1em}
  \begin{center}
  \begin{minipage}{0.87\textwidth} 
  \textbf{Palavras-chave:} value investing, valor intrínseco, Benjamin Graham.
  \end{minipage}
  \end{center}
  \clearpage
}

\begin{meuresumo}
Este projeto busca preencher uma lacuna identificada na literatura brasileira ao avaliar empiricamente a eficácia da fórmula de valor intrínseco proposta por Benjamin Graham como critério adicional para a seleção e o rebalanceamento de carteiras de ações no mercado nacional. A metodologia adota essa fórmula como parâmetro complementar aos filtros fundamentais adaptados por Palazzo et al. (2018), sendo utilizada tanto para o momento de entrada — quando o preço de mercado estiver inferior a dois terços do valor intrínseco estimado — quanto para o ponto de saída, com base na margem de segurança residual. Para isso, serão construídas carteiras teóricas com ações de empresas listadas na B3, com base em dados fundamentalistas do período de 2014 a 2024. O desempenho das estratégias será comparado em diferentes protocolos de rebalanceamento (anual, trimestral) e  buy-and-hold, e avaliado por métricas de retorno ajustado ao risco, como o alfa de Jensen e o índice de Sharpe.
\end{meuresumo}




% formatação do título de cada seção
\makeatletter
\def\@makechapterhead#1{%
  \vspace*{50\p@}%
  {\parindent \z@ \raggedright \normalfont
    \ifnum \c@secnumdepth >\m@ne
      \huge\bfseries \thechapter\space
    \fi
    \huge \bfseries #1\par\nobreak
    \vskip 40\p@
  }}
\makeatother


