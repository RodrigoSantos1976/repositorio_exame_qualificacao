
\thispagestyle{empty}

\begin{center}
    \small
    \sc Instituto Federal de Educação, Ciência e Tecnologia de Minas Gerais \\ 
    \sc Campus Formiga \\
    \sc Mestrado Profissional em Administração
    
   
    \vspace{3cm}
    \large {Rodrigo Emiliano dos Santos}
    
    \vspace{3cm}
     \large \textbf{Value Investing no Mercado Brasileiro} \\
    Uma análise comparativa entre os Filtros de Graham e uma nova métrica de Valor Intrínseco com Aplicação Educacional
    
    \vspace{2cm}
    \begin{flushright}
    \begin{minipage}{0.6\textwidth}
    \small
    Projeto apresentado ao Programa de Pós-
    Graduação em Administração do Instituto
    Federal de Educação, Ciência e
    Tecnologia de Minas Gerais - \textit{Campus} Formiga 
    (IFMG - \textit{Campus} Formiga), como requisito para 
    o Exame de Qualificação do Mestrado Profissional em 
    Administração.
    \end{minipage}
    \end{flushright}
    
    \vspace{0.5cm}
    \begin{flushright}
    \small
    Orientador: Prof. Dra Maisa Kely de Melo.\\
    Coorientador: Prof. Dr. Washighton Santos Silva.
    
    \vspace{0.5cm}
    Linha de Pesquisa: Finanças Corporativas e Investimentos.
    \end{flushright}
    
    \vfill
    Formiga, Minas Gerais \\
    2025
\end{center}


\newpage

\thispagestyle{empty}

\newenvironment{meuresumo}{
  \clearpage
  \small
  \vspace{-1cm}
  \begin{center}
    \bfseries RESUMO
    \vspace{0.5em}
  \end{center}
  \begin{quote}
}{
  \end{quote}
  \vspace{-1.1em}
  \begin{center}
  \begin{minipage}{0.87\textwidth} 
  \textbf{Palavras-chave:} value investing, valor intrínseco, Benjamin Graham.
  \end{minipage}
  \end{center}
  \clearpage
}

\begin{meuresumo}
Este estudo investiga a eficácia de estratégias de value investing no mercado acionário brasileiro, com foco na comparaççao entre os filtros clássicos adaptados por Pallazo (2018) e uma nova métrica baseada na fórmula de valor intrínseco proposta por Benjamin Graham. Utilizando dados fundamentalistas de empresas listadas na B3, são construídas carteiras de investimento que serão avaliadas em termos de retorno, risco e desempenho ajustado. Como contribuição prática, propõe-se o desenvolvimento de um aplicativo educacional que permitirá a simulação e o acompanhamento destas carteiras, promovendo a integração entre teoria financeira e prática pedagógica. 
\end{meuresumo}




% formatação do título de cada seção
\makeatletter
\def\@makechapterhead#1{%
  \vspace*{50\p@}%
  {\parindent \z@ \raggedright \normalfont
    \ifnum \c@secnumdepth >\m@ne
      \huge\bfseries \thechapter\space
    \fi
    \huge \bfseries #1\par\nobreak
    \vskip 40\p@
  }}
\makeatother


